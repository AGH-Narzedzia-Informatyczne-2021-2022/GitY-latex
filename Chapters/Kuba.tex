\newpage
\section{Jakub Banach}
\label{sec:skoki}
\subsection{Skoki narciarskie}
    
    \hspace{0.5cm}\textbf{Skoki narciarskie} - to zimowa dyscyplina sportowa rozgrywana na skoczniach narciarskich już od połowy XIX wieku. Wraz z biegami narciarskimi i kombinacją norweską \emph{(biegi + skoki)} tworzą rodzinę sportów narciarstwa klasycznego. Dyscyplina ta cieszy się ogromną popularnością w Europie, szczególnie w krajach nordyckich \textbf{(Norwegia i Finlandia)}, Europie Środkowej \textbf{(Austria, Czechy, Niemcy, Polska, Słowenia czy Szwajcaria)}, a poza Europą głównie w \textbf{Japonii}. \par
    
    Celem jest wykonanie jak najdłuższego skoku po rozpędzeniu się i odbiciu od progu skoczni. Na największych skoczniach, tzw. mamucich, możliwe są skoki przekraczające 250 metrów \emph{(konkurencje te nazywa się wtedy lotami narciarskimi)}. \textbf{Ocenia się odległość uzyskaną przez zawodnika, styl skoku oraz punkty za wiatr}. \par

\subsection{Wzór na przeliczanie punktów za wiatr}
    \begin{center}
    $\Delta w = TWG \frac{(HS-36)}{20}$ 
    \end{center}
    \begin{center}
    \emph{\textbf{HS} (hill size) - rozmiar skoczni [metry]} \par
    \emph{\textbf{TWG} - tangensowa predkość wiatru (średnia wartość [m/s])}\par
    \emph{\textbf{$\Delta $w} - wpływ wiatru na odległość skoku[m]}\par
    \end{center}

\subsection{Zdjęcie}
\graphicspath{ {./Pictures/} }
    \begin{figure}[ht]
    \centering
    \includegraphics[scale=0.8]{Pictures/skoczek.png}
    \caption{\emph{Skoczek w trakcie lotu}}
    \label{fig:skoczek}
    \end{figure}

\subsection{Punktowanie pierwszych 10 miejsc}
    \begin{table}[h!]
    \centering
    \begin{tabular}{|c|c|}
    \hline
    \textbf{Miejsce} & \textbf{Liczba punktów} \\ \hline
    1                & 100                     \\ \hline
    2                & 80                      \\ \hline
    3                & 60                      \\ \hline
    4                & 50                      \\ \hline
    5                & 45                      \\ \hline
    6                & 40                      \\ \hline
    7                & 36                      \\ \hline
    8                & 32                      \\ \hline
    9                & 29                      \\ \hline
    10               & 26                      \\ \hline
    \end{tabular}
    \caption{Ilość punktów w zależności od miejsca}
    \label{table:punkty}
\end{table}

\subsection{Najlepsze kraje zeszłego sezonu}
    \begin {enumerate} 
        \item Norwegia
        \item Polska
        \item Niemcy
     \end {enumerate}
 
\subsection{Skład reprezentacji Polski zeszłego sezonu}
    \begin{itemize}
        \item[!] Kamil Stoch
        \item[!] Dawid Kubacki
        \item[!] Piotr Żyła
        \item[*] Jakub Wolny
        \item[?] Stefan Hula
        \item[*] Klemens Murańka
        \item[?] Paweł Wąsek
        \item[?] Tomasz Pilch
        \item[*] Aleksander Zniszczoł
        \item[!] Andrzej Stękała
        \item[?] Maciej Kot
    \end{itemize}