%\documentclass[16pt, letterpaper]{article}
%\usepackage[utf8]{inputenc}
%\usepackage{graphicx}
%\graphicspath{ {./Pictures/} }
%\usepackage{indentfirst} %??????

% \title{My first project in \LaTeX{}}
% \section{Kaja Dzielnicka}
% \date{}
% \maketitle
\newpage
\section{Kaja Dzielnicka}
\label{sec:kajadzielnicka}
\subsection{Equation}
    $$ \frac{a}{sin\alpha}=\frac{b}{sin\beta}=\frac{c}{sin\gamma}=2R$$

\subsection{Photo}
    \begin{figure}[h]
        \centering
        \includegraphics[width=0.35\textwidth]{./Pictures/kitty}
        \caption{British Shorthair.}
        \label{fig:cat}
    \end{figure}
    
\subsection{Table}
    \begin{table}[h!]
    \centering
    \begin{tabular}{|c | c c c c|} 
     \hline
         & 5 lbs & 10 lbs & 15 lbs & 20 lbs \\ [0.5ex] 
     \hline\hline
        Kitten & 200 kcal & 400 kcal & 600 kcal & 800 kcal \\ 
        Lean Cat & 170 kcal & 280 kcal & 360 kcal & 440 kcal \\
        Overweight Cat & 180 kcal & 240 kcal & 280 kcal & 310 kcal \\
        Nursing/Pregnant Cat & 336 kcal & 603 kcal & 851 kcal & 1091 kcal \\
    \hline
    \end{tabular}
    \caption{How much wet food to feed a cat every day}
    \label{table:food}
\end{table}
    
\subsection{Numbered (ordered) list}
    Basic facts:
    \begin{enumerate}
        \item The British Shorthair is a domesticated cat.
        \item Its features make it a popular breed in cat shows.
        \item It has been the most popular breed of cat registered by the UK's Governing Council of the Cat Fancy (GCCF) since 2001 when it overtook the Persian breed.
    \end{enumerate}
    
\subsection{Bulleted (unordered) list}
    Body characteristics:
    \begin{itemize}
         \item British shorthairs have dense, plush coats that are often described as crisp or cracking, which refers to the way the coat breaks over the cat's body contours.
         \item[!] Eyes are large, round and widely set. They can be a variety of colours, though the copper or gold eyes of the British cream are the best known
        \item[ ] They have round heads with full, chubby cheeks and a body that is rounded and sturdy.
        \item [*] British shorthairs are large and muscular, and are described as having a cobby build.
    \end{itemize}
    
\subsection{Short text}
    British shorthairs are an easygoing breed of cat. They have a \underline{stable character} and take well to being kept as \underline{indoor-only} cats, making them \textbf{ideal} for apartment living. They are not terribly demanding of attention, though they will let their owner know if they feel like playing. They are \underline{not hyperactive} cats, preferring to sit close to their owners rather than on them. They might supervise household activities from a comfortable perch or perhaps the floor. \par
    \textbf{British shorthairs are \emph{wonderful} cats for people who work}, as they are very happy to simply laze around the house while their owner is out. They \emph{do not get destructive} or need other animals for company, though they do enjoy having another British shorthair or a cat with similar temperament around. \par
    As you can see in the figure \ref{fig:cat}, these cats are very beautiful. \par
    Table \ref{table:food} shows how much should they eat.
